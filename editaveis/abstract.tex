\begin{resumo}[Abstract]
 \begin{otherlanguage*}{english}
   The difficulty of Brazilian elementary school students have in understanding the basic math curriculum is not new to teachers. For this reason, various pedagogical methods are being used to improve the retention of the content taught. Serious games emerged as a play based methodology to increase students' retention of the math curriculum, but the fun aspect, which is responsible for providing players' enjoyment is often overlooked by the genre. This paper uses concepts from game software engineering and traditional software engineering to create the game \textit{Math Hero}, to help teach mental math techniques to Brazillian elementary school students. Based on the aforementioned concepts, a alpha version of the game was implemented that contains a mixture of two game modes time attack programmed for the final development, the competition and training mode, so that it would be possible to collect, analyze and compare gameplay data between players with and without previous instruction, using the ``Mental Math Tournament'' event that took place during the University Extension Week 2024 at the University of Brasilia.
   \vspace{\onelineskip}
 
   \noindent 
   \textbf{Key-words}: Game Engineering. Software Engineering. Games. Mental Math. Mathematics.
 \end{otherlanguage*}
\end{resumo}