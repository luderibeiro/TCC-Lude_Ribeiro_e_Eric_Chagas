\begin{resumo}
 A dificuldade de alunos do ensino básico brasileiro na compreensão do currículo base de matemática não é novidade para os professores. Por isso diversos métodos pedagógicos vêm sendo utilizados a fim de melhorar a fixação dos conteúdos ensinados. Os \textit{serious games} surgiram como proposta de metodologia lúdica para aumentar a fixação dos conteúdos por parte dos alunos, porém, o aspecto lúdico, que é responsável por proporcionar o divertimento dos jogadores, muitas vezes é deixado de lado. Este trabalho utiliza os conceitos da engenharia de \textit{games} e da engenharia de software tradicional para a criação do \textit{game} Math Hero, para auxiliar no ensino de técnicas de cálculo mental para alunos do ensino básico brasileiro. Tendo como base os conceitos citados, foi implementada uma versão \textit{alpha} do jogo que contém uma mistura entre duas modalidades de modo de jogo \textit{time attack} programadas para o desenvolvimento final, a modalidade competição e de treino, para que fosse possível a coleta, a análise e a comparação de dados de jogabilidade entre jogadores com e sem instrução prévia, no evento ``Torneio de matemática mental'' ocorrido durante a Semana de Extensão Universitária 2024 da Universidade de Brasília.
 \vspace{\onelineskip}
    
 \noindent
 \textbf{Palavras-chave}: Engenharia de \textit{Games}. Engenharia de Software. \textit{Games}. Cálculo Mental. Matemática.
\end{resumo}
