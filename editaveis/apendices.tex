\begin{apendicesenv}

% ===========================================================================
% APÊNDICE A: MAPEAMENTO ISO/IEC 25010 (Manutenibilidade)
% ===========================================================================
\chapter{Mapeamento das Métricas de Manutenibilidade}
\label{apendice:map_manut}

\begin{table}[H]
    \centering
    \scriptsize
    \caption{Relação entre as métricas de Manutenibilidade do estudo e as subcaracterísticas da norma ISO/IEC 25010:2023 \cite{isoiec25010}.}
    \label{tab:apendice_manutenibilidade}
    \begin{tabularx}{\textwidth}{l X X}
        \toprule
        \textbf{Subcaracterística} & \textbf{Métrica (SonarQube)} & \textbf{Relação e Alinhamento} \\
        \midrule
        \textbf{Analisabilidade} & \textbf{Dívida Técnica} & Medida em tempo de remediação, indica o esforço para entender e consertar o código. Alto nível de dívida compromete a analisabilidade. \\
        \cmidrule{2-3}
        & \textbf{Code Smells} & Tornam o código complexo de diagnosticar. A redução melhora a facilidade de análise. \\
        \midrule
        \textbf{Modificabilidade} & \textbf{Complexidade Ciclomática} & Funções complexas são difíceis de modificar. O controle garante modificações seguras. \\
        \cmidrule{2-3}
        & \textbf{Código Duplicado} & Exige mudanças em múltiplos locais, aumentando risco de erros. \\
        \midrule
        \textbf{Testabilidade} & (Cobertura de Testes) & Embora crítica, a cobertura complementa a análise de manutenibilidade. \\
        \bottomrule
    \end{tabularx}
\end{table}

% ===========================================================================
% APÊNDICE B: MAPEAMENTO ISO/IEC 25010 (Confiabilidade)
% ===========================================================================
\chapter{Mapeamento da Métrica de Confiabilidade}
\label{apendice:map_conf}

\begin{table}[H]
    \centering
    \scriptsize
    \caption{Relação entre a métrica de Confiabilidade do estudo e as subcaracterísticas da norma ISO/IEC 25010:2023 \cite{isoiec25010}.}
    \label{tab:apendice_confiabilidade}
    \begin{tabularx}{\textwidth}{l X X}
        \toprule
        \textbf{Subcaracterística} & \textbf{Métrica (SonarQube)} & \textbf{Relação e Alinhamento} \\
        \midrule
        \textbf{Tolerância a Falhas} & \textbf{Bugs} & Bugs são a manifestação direta de problemas de confiabilidade. Sua redução aumenta a confiança no sistema. \\
        \bottomrule
    \end{tabularx}
\end{table}

% ===========================================================================
% APÊNDICE C: MAPEAMENTO ISO/IEC 25010 (Segurança)
% ===========================================================================
\chapter{Mapeamento das Métricas de Segurança}
\label{apendice:map_seg}

\begin{table}[H]
    \centering
    \scriptsize
    \caption{Relação entre as métricas de Segurança do estudo e as subcaracterísticas da norma ISO/IEC 25010:2023 \cite{isoiec25010}.}
    \label{tab:apendice_seguranca}
    \begin{tabularx}{\textwidth}{l X X}
        \toprule
        \textbf{Subcaracterística} & \textbf{Métrica (SonarQube)} & \textbf{Relação e Alinhamento} \\
        \midrule
        \textbf{Integridade} & \textbf{Vulnerabilidades} & Correção de falhas que permitem exploração do sistema. \\
        \cmidrule{2-3}
        \textbf{Confidencialidade} & \textbf{Hotspots de Segurança} & Pontos sensíveis que exigem revisão manual para garantir proteção de dados. \\
        \bottomrule
    \end{tabularx}
\end{table}

% ===========================================================================
% APÊNDICE D: PROTOCOLO DE REPRODUCIBILIDADE
% ===========================================================================
\chapter{Protocolo de Reproducibilidade Experimental}
\label{apendice:protocolo_reproducibilidade}

Este apêndice documenta os artefatos de configuração utilizados. O objetivo é permitir a reprodução do experimento, contrastando a complexidade do projeto inicial (Zulip) com a solução otimizada (Django Base).

\section{Ambiente 1: Projeto Zulip (Fase Exploratória)}
A complexidade da infraestrutura de testes do Zulip inviabilizou o ciclo de feedback rápido. Abaixo, o trecho do arquivo \texttt{zulip-ci.yml} que demonstra a dependência de contêineres customizados e scripts proprietários.

\begin{verbatim}
jobs:
  tests:
    strategy:
      matrix:
        include:
          - docker_image: zulip/ci:jammy
            name: Ubuntu 22.04 (Python 3.10)
    container: ${{ matrix.docker_image }}
    steps:
      - name: Install dependencies
        run: |
          ./tools/ci/setup-backend --skip-dev-db-build
      - name: Run backend tests
        run: |
          ./tools/test-backend --coverage
\end{verbatim}

\section{Ambiente 2: Projeto Django Base (Ambiente Ativo)}
Ambiente onde as intervenções foram executadas. Utiliza padrões de mercado para garantir reprodutibilidade.

\subsection{Containerização (Docker Compose)}
Trecho do arquivo \texttt{docker-compose.dev.yml} utilizado:

\begin{verbatim}
services:
    project:
        container_name: project_dev
        build:
            context: .
            dockerfile: Dockerfile.dev
        ports:
            - "8000:8000"
        depends_on:
            project_db:
                condition: service_healthy

    project_db:
        image: postgres:13-alpine
        healthcheck:
            test: ["CMD-SHELL", "pg_isready -U postgres"]
            interval: 5s
\end{verbatim}

\subsection{Automação (Makefile)}
Comandos utilizados para padronizar a execução dos testes e análises.
\textit{Nota: Os emojis originais foram removidos para compatibilidade com a compilação do documento.}

\begin{verbatim}
# Execucao da suite de testes completa
test:
    @echo "[TEST] Executando testes..."
    @cd $(PROJECT_DIR) && . ../$(VENV)/bin/activate && \
    export PYTHONPATH=$$PWD && $(PYTEST) -v core cart

# Medicao de cobertura para o Sonar
test-coverage:
    @echo "[COVERAGE] Executando testes com cobertura..."
    @cd $(PROJECT_DIR) && . ../$(VENV)/bin/activate && \
    export PYTHONPATH=$$PWD && \
    $(PYTEST) --cov=core --cov=cart --cov-report=xml

# Analise de Seguranca
security-check:
    @echo "[SECURITY] Verificando vulnerabilidades..."
    @cd $(PROJECT_DIR) && . ../$(VENV)/bin/activate && pip-audit
\end{verbatim}

\subsection{Pipeline CI/CD (GitHub Actions)}
Configuração do \textit{workflow} para validação contínua.

\begin{verbatim}
name: CI/CD Pipeline
on: [push, pull_request]
jobs:
  test-and-quality:
    runs-on: ubuntu-latest
    steps:
      - name: Set up Python
        uses: actions/setup-python@v4
        with:
          python-version: "3.12"
      - name: Run tests
        run: |
          cd project
          pytest --cov=core --cov=cart --cov-report=xml:coverage.xml
      - name: SonarCloud Scan
        uses: SonarSource/sonarcloud-github-action@master
\end{verbatim}

% ===========================================================================
% APÊNDICE E: DIÁRIO DE BORDO (EXPANDIDO)
% ===========================================================================
\chapter{Diário de Bordo Experimental}
\label{apendice:diario_bordo}

Este apêndice apresenta o registro cronológico e detalhado das sessões de pareamento com as ferramentas de IA. O objetivo é documentar não apenas o código final, mas o processo de construção, incluindo as falhas de contexto, as alucinações da ferramenta e as intervenções humanas corretivas.

\section*{Entrada 1: Definição Arquitetural e Configuração Inicial}
\label{diario:entrada1}
\textbf{Data:} 24/11/2025

\begin{table}[H]
    \centering
    \small
    \begin{tabularx}{\textwidth}{|l|X|}
        \hline
        \textbf{Pesquisador} & Lude Ribeiro \\ \hline
        \textbf{Requisito} & Arquitetura do Módulo (Modular Monolith) \\ \hline
        \textbf{Ferramenta} & ChatGPT 4.0 (Web) \\ \hline
        \textbf{Tempo} & aproximadamente 4 horas \\ \hline
    \end{tabularx}
\end{table}

\textbf{1. Objetivo:} Definir a estrutura de pastas para o novo módulo \texttt{cart} respeitando a Clean Architecture já existente.

\textbf{2. Interação (Prompt):}
\begin{quote}
"Atue como um Arquiteto de Software. Preciso criar um NOVO módulo isolado chamado \texttt{cart}. Forneça um script bash para criar a árvore de diretórios e a configuração do \texttt{apps.py}."
\end{quote}

\textbf{3. Resultado e Análise:}
A IA gerou um script tecnicamente correto, mas posicionalmente errado. Ela assumiu a raiz do repositório git como raiz do projeto Django.
\begin{itemize}
    \item \textbf{Falha:} O script criaria pastas fora de \texttt{project/}.
    \item \textbf{Intervenção Humana:} Execução manual do script dentro do diretório correto.
    \item \textbf{Conclusão:} A IA carece de "consciência espacial" do projeto sem acesso direto à árvore de arquivos.
\end{itemize}

\section*{Entrada 2: A Crise de Infraestrutura e Perda de Contexto}
\label{diario:entrada2}
\textbf{Data:} 25/11/2025

\begin{table}[H]
    \centering
    \small
    \begin{tabularx}{\textwidth}{|l|X|}
        \hline
        \textbf{Requisito} & Configuração de Ambiente (Docker) \\ \hline
        \textbf{Ferramenta} & ChatGPT 4.0 (Web) \\ \hline
    \end{tabularx}
\end{table}

\textbf{1. Incidente:} Durante o setup, ocorreu um erro de rede no Docker (\texttt{IPv6 network unreachable}).
\textbf{2. Impacto no Experimento:} A sessão de debugging consumiu mais de 40 interações. Ao tentar retornar ao código Python, a IA sofreu \textbf{Saturação de Contexto}, "esquecendo" as definições arquiteturais feitas na Entrada 1 e sugerindo código fora do padrão estabelecido.

\section*{Entrada 3: Falha na Implementação da API e Pivô de Ferramenta}
\label{diario:entrada3}
\textbf{Data:} 26/11/2025

\begin{table}[H]
    \centering
    \small
    \begin{tabularx}{\textwidth}{|l|X|}
        \hline
        \textbf{Requisito} & Implementação de Views e Serializers \\ \hline
        \textbf{Ferramenta} & ChatGPT (Web) $\rightarrow$ \textbf{Migração para GitHub Copilot} \\ \hline
    \end{tabularx}
\end{table}

\textbf{1. Problema (Loop de Erro):}
Ao tentar implementar o \texttt{CartViewSet}, o ChatGPT gerou um código que resultava em \texttt{400 Bad Request}. Ao ser confrontado com o erro, a ferramenta entrou em contradição, sugerindo alterações cíclicas entre \texttt{ModelViewSet} e \texttt{GenericViewSet} sem resolver a causa raiz (validação do serializer).

\textbf{2. Decisão Metodológica (Pivot):}
Diagnosticou-se que a ferramenta baseada em chat era insuficiente para refatoração de múltiplos arquivos interdependentes. Optou-se pela migração para o \textbf{GitHub Copilot Agent} no VS Code, utilizando a flag \texttt{@workspace} para leitura de contexto local.

\section*{Entrada 4: Implementação de Lógica Complexa (Service)}
\label{diario:entrada4}
\textbf{Data:} 27/11/2025

\begin{table}[H]
    \centering
    \small
    \begin{tabularx}{\textwidth}{|l|X|}
        \hline
        \textbf{Ferramenta} & GitHub Copilot Agent \\ \hline
        \textbf{Requisito} & Regras de Negócio: Estoque e Atomicidade \\ \hline
    \end{tabularx}
\end{table}

\textbf{1. Prompt Técnico:}
\begin{quote}
"@workspace Implemente a lógica de \texttt{add\_item} no \texttt{CartService}. Requisitos: Validação de estoque (lançar ValueError), Snapshot de preço e garantia de atomicidade de banco de dados."
\end{quote}

\textbf{2. Resultado (Sucesso):}
A IA gerou uma implementação robusta:
\begin{itemize}
    \item Usou \texttt{transaction.atomic} e \texttt{select\_for\_update} corretamente para evitar \textit{Race Conditions}.
    \item Sugeriu proativamente a extração da validação para um método privado \texttt{\_validate\_stock}, reduzindo a complexidade ciclomática (Atendendo RNF-A2).
\end{itemize}

\section*{Entrada 5: Testes Automatizados e Correção de Configuração}
\label{diario:entrada5}
\textbf{Data:} 28/11/2025

\begin{table}[H]
    \centering
    \small
    \begin{tabularx}{\textwidth}{|l|X|}
        \hline
        \textbf{Requisito} & RNF-A1 (Cobertura de Testes) \\ \hline
        \textbf{Resultado} & \textbf{92\% de Cobertura} no Módulo Cart \\ \hline
    \end{tabularx}
\end{table}

\textbf{1. Incidente de Configuração:}
Ao rodar os testes gerados, ocorreu o erro \texttt{RuntimeError: Model class ... isn't in INSTALLED\_APPS}.
\textbf{2. Correção Assistida:}
O Copilot analisou o workspace e identificou que, embora o app estivesse no \texttt{settings.py} principal, ele estava ausente do \texttt{settings\_test.py} usado pelo Pytest. A IA gerou o patch de correção imediatamente.

\section*{Entrada 6: Refinamento de API (Correção de URLs)}
\label{diario:entrada6}
\textbf{Data:} 09/12/2025

\begin{table}[H]
    \centering
    \small
    \begin{tabularx}{\textwidth}{|l|X|}
        \hline
        \textbf{Requisito} & UX da API e Padrões REST \\ \hline
        \textbf{Problema} & Redundância nas rotas (\texttt{/cart/v1/cart/...}) \\ \hline
    \end{tabularx}
\end{table}

\textbf{1. Análise Humana:}
A inspeção do Swagger revelou URLs mal formadas devido à configuração padrão do \texttt{DefaultRouter} do DRF sugerida pela IA.

\textbf{2. Ação Corretiva:}
Solicitou-se a correção do \texttt{urls.py} para registrar a rota base como vazia, normalizando os endpoints para \texttt{/cart/v1/add\_item/}. Isso exigiu a atualização subsequente de todos os testes de integração que quebraram (Regressão).

\section*{Entrada 7: Refinamento Semântico (Remoção de Paginação)}
\label{diario:entrada7}
\textbf{Data:} 10/12/2025

\begin{table}[H]
    \centering
    \small
    \begin{tabularx}{\textwidth}{|l|X|}
        \hline
        \textbf{Requisito} & Semântica de Negócio (Carrinho Único) \\ \hline
        \textbf{Problema} & Endpoint GET retornava lista paginada \\ \hline
    \end{tabularx}
\end{table}

\textbf{1. Problema Semântico:}
A IA aplicou o padrão genérico de "Listagem" do Django Rest Framework, retornando uma estrutura paginada (`count`, `results: [...]`) para o endpoint de visualização do carrinho.
\textbf{2. Intervenção Humana:}
O pesquisador identificou que, semanticamente, um usuário possui apenas \textbf{um} carrinho ativo. Foi solicitado à IA que sobrescrevesse o método \texttt{list} do ViewSet para retornar o objeto direto, simplificando o consumo pelo frontend.

\textbf{3. Conclusão da Intervenção A:}
Com este ajuste final, o módulo atingiu os critérios de qualidade funcional, semântica e de cobertura, encerrando-se a fase de desenvolvimento do Carrinho.

\end{apendicesenv}