\chapter{Introdução}\label{chapter:introducao}

\section{Contexto}

\subsection{A nova realidade}

\begin{citacao}
``A IA generativa é a tecnologia mais transformadora do nosso tempo, possibilitando avanços em criatividade, produtividade e descoberta científica'' \citeonline{huang_2023}.
\end{citacao}

Desde novembro de 2022, o campo da tecnologia testemunha uma transformação significativa com a ascensão da Inteligência Artificial (IA) generativa, impulsionada pelo lançamento em larga escala do ChatGPT pela OpenAI. Essa ferramenta marcou o início de uma nova era no desenvolvimento de software, redefinindo processos e práticas tradicionais.

Nesse contexto de ``\textit{nova realidade}'', emergem desafios associados a cenários antes considerados distópicos ou restritos à ficção científica. A IA tem transformado profundamente o desenvolvimento de software, introduzindo ferramentas que automatizam e aprimoram processos outrora exclusivos de desenvolvedores humanos. Ferramentas como GitHub Copilot, ChatGPT e Claude, baseadas em IA generativa, permitem a geração de código, sugestões em tempo real e execução de tarefas complexas a partir de comandos em linguagem natural, os chamados \textit{prompts}. Esse cenário redefine a interação entre desenvolvedores e sistemas computacionais, inaugurando um ambiente colaborativo entre humanos e máquinas.

\begin{citacao}
``Eu diria que talvez 20\%, 30\% do código que está em nossos repositórios hoje e em alguns de nossos projetos provavelmente foi todo escrito por software'' \citeonline{nadella_2025}\footnote{As datas futuras em referências de notícias e artigos de opinião refletem a natureza prospectiva e de rápida evolução do tema, posicionando a discussão no contexto do ano de defesa desta monografia.}.
\end{citacao}

A adoção cada vez mais acelerada dessas ferramentas por empresas e comunidades de desenvolvedores levanta questões sobre produtividade, qualidade, segurança e o papel do desenvolvedor no processo de produção de código. Relatórios corporativos, como os da Microsoft e do GitHub, indicam que uma proporção significativa de código já é gerada por IA, evidenciando a escala dessa transformação. Contudo, os impactos reais dessas tecnologias — incluindo benefícios, desafios e riscos — ainda carecem de análises sistemáticas baseadas em dados públicos.

\begin{citacao}
``Como em todas as revoluções tecnológicas, espero que haja um impacto significativo nos empregos, mas é muito difícil prever exatamente como será esse impacto'' \citeonline{altman_2023}.
\end{citacao}

Esta pesquisa é motivada pela necessidade de compreender como a IA está moldando o desenvolvimento de software, tanto em termos técnicos quanto nas dinâmicas de trabalho. A disponibilidade crescente, embora limitada, de dados públicos — como relatórios de empresas e repositórios — oferece uma oportunidade de investigar essas mudanças e contribuir para o debate acadêmico e profissional sobre o futuro da programação.

\subsection{Motivação dos autores}

Como estudantes de Engenharia de Software, vivenciamos um momento de ruptura tecnológica que redefine nossa profissão e o ecossistema ao seu redor. A integração acelerada da IA no desenvolvimento de software suscita dúvidas, expectativas e receios que nos motivam a investigar como tais ferramentas estão transformando os processos de produção de código e o papel dos desenvolvedores.

\begin{citacao}
``Nos próximos 18 meses, a maior parte da codificação será feita por IA, e ela será melhor que o trabalho da maioria dos engenheiros plenos'' \citeonline{zuckerberg_2025}.
\end{citacao}

Declarações como essa, feitas por figuras públicas com grande influência no setor tecnológico, intensificam questionamentos sobre o futuro do trabalho em programação. De modo semelhante, Elon Musk apresenta uma visão ainda mais radical:

\begin{citacao}
``Provavelmente, nenhum de nós terá um emprego [...] Haverá um ponto em que nenhum trabalho será necessário. Você poderá ter um trabalho se quiser ter um trabalho por satisfação pessoal, mas a IA e os robôs fornecerão todos os bens e serviços que você desejar'' \citeonline[tradução nossa]{musk_2024_vivatech}.
\end{citacao}

Embora inseridas em um discurso de ``abundância universal'', tais afirmações funcionam como catalisadores de incertezas, especialmente para estudantes e profissionais em formação.

Além dessas projeções, notícias recentes abordam as dificuldades enfrentadas por profissionais de software no uso de ferramentas de IA, desde iniciantes até desenvolvedores experientes \citeonline{g1_2024}. As percepções sobre o futuro variam entre visões pessimistas e otimistas. Entre os pontos positivistas está a possível diminuição da necessidade do trabalho sistemático de escrita de código.

Segundo Bianchin (2025), Jensen Huang argumenta que não é mais necessário aprender a programar, pois a IA já pode assumir essa função. Nessa perspectiva, engenheiros de software poderiam dedicar mais tempo a atividades que exigem criatividade e inovação \citeonline{xataka_2025}.

Contribuindo com essa visão, McFarland discute o conceito de \textit{Vibe Coding}, no qual modelos de linguagem permitem que qualquer pessoa crie aplicações ao simplesmente descrever ideias em linguagem natural. Ferramentas como Cursor e Lovable integram agentes inteligentes que ampliam o público capaz de desenvolver software \citeonline{mcfarland2025vibe}.

Entretanto, há também visões críticas. Bianchin evidencia preocupações de programadores experientes com a crescente dependência das ferramentas de IA. Para Namanyay Goel, desenvolvedores iniciantes produzem mais código, porém compreendem menos os fundamentos de suas soluções — o que pode gerar consequências futuras \citeonline{xataka_2025}.

Esse cenário demanda mais do que adoção tecnológica: exige transformação estratégica e cultural. Como ressalta \citeonline{salvador_2024}, o uso da IA deve integrar-se cedo ao fluxo de trabalho, sob metas claras, governança robusta de dados, capacitação contínua e estratégias iterativas de implementação.

Questões jurídicas também emergem. Conforme Basilio \cite{basilio_2025}, especialistas divergem sobre a responsabilidade por danos causados por agentes de IA. Empresas podem ser responsabilizadas mesmo quando o erro não se origina nelas, enquanto representantes das desenvolvedoras argumentam que usuários devem assumir parte da responsabilidade ao serem informados sobre limitações dos sistemas. Problemas como ``superengenharia'' — a criação de arquiteturas excessivamente complexas com múltiplos agentes — também são apontados como riscos \cite{basilio_2025}.

Esse conjunto de incertezas motiva a pergunta central deste trabalho:

\begin{citacao}
\textit{Quais são os reais impactos da Inteligência Artificial na produção de código e como eles afetam o futuro do desenvolvimento de software?}
\end{citacao}

Com isso, propomos uma análise crítica dos efeitos da IA, investigando suas aplicações e implicações para produtividade, qualidade e relações de trabalho.

\section{Problema}\label{section:problema}

A rápida adoção de ferramentas de IA no desenvolvimento de software não tem sido acompanhada, de maneira suficiente, por estudos sistemáticos que avaliem seus impactos reais no processo de produção de código. Embora prometam maior produtividade e automação, persistem incertezas quanto à qualidade, segurança, manutenibilidade e implicações éticas e profissionais. Assim, torna-se necessária uma investigação baseada em métricas objetivas e dados públicos (ou privados, nos casos de estudos controlados).

\section{Objetivos}\label{section:objetivos}

O objetivo geral deste trabalho é avaliar o impacto prático do uso de ferramentas de IA generativa — especificamente o GitHub Copilot e o ChatGPT — na manutenção e evolução de software. Para isso, combina-se uma revisão teórica focada com um estudo de caso empírico realizado no projeto de código aberto \textit{Zulip}, no qual atividades de refatoração e desenvolvimento incremental são analisadas quantitativamente a partir da ferramenta de análise estática SonarQube.

Para atingir esse objetivo, foram definidos os seguintes objetivos específicos:

\begin{itemize}
\item \textbf{Sintetizar}, por meio de revisão da literatura, as principais discussões, métricas e desafios sobre IA, qualidade e produtividade no desenvolvimento de software.
\item \textbf{Estabelecer uma linha de base} de qualidade para um componente selecionado do projeto \textit{Zulip}, por meio de análise inicial com o SonarQube.
\item \textbf{Executar} um ciclo de manutenção e desenvolvimento empregando o GitHub Copilot como assistente de codificação e o ChatGPT como ferramenta de geração de blocos de código e estratégias de refatoração.
\item \textbf{Avaliar comparativamente} os artefatos de código antes e após a intervenção assistida por IA, quantificando variações em métricas de qualidade, segurança e dívida técnica.
\item \textbf{Discutir} as implicações dos resultados à luz da literatura, analisando a eficácia, os benefícios e os desafios das ferramentas avaliadas.
\end{itemize}

\section{Contribuições Esperadas}\label{sec:contribuicoes}

Espera-se que este trabalho contribua para a comunidade acadêmica e profissional em três dimensões principais. Primeiro, fornecendo uma \textbf{análise quantitativa} do impacto de ferramentas de IA na qualidade de um software complexo e ativo. Segundo, apresentando um \textbf{protocolo de medição replicável} para futuros estudos. Por fim, oferecendo \textbf{insights qualitativos} sobre a colaboração humano–IA durante tarefas de refatoração e desenvolvimento, a partir de registros sistemáticos do processo.

\section{Organização do Trabalho}\label{section:organizacao}

% Este trabalho está estruturado em cinco capítulos. A \hyperref[chapter:introducao]{Introdução} apresenta o contexto e a motivação. O \hyperref[chapter:referencial_teorico]{Referencial Teórico} discute conceitos, ferramentas e estudos relevantes. A \hyperref[chapter:Metodologia]{Metodologia} descreve os métodos empregados. Os \hyperref[chapter:Resultados]{Resultados} trazem análises e dados coletados. Por fim, as \hyperref{chapter:conclusao}{Considerações Finais} sintetizam as conclusões, limitações e direções futuras.
