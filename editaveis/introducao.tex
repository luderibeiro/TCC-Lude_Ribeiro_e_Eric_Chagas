\chapter{Introdução}\label{chapter:introducao}

\section{Contexto}

\subsection{A nova realidade}

\begin{citacao}
``A IA generativa é a tecnologia mais transformadora do nosso tempo, possibilitando avanços em criatividade, produtividade e descoberta científica'' \citeonline{huang_2023}.
\end{citacao}

Desde novembro de 2022, o campo da tecnologia testemunha uma transformação significativa com a ascensão da Inteligência Artificial (IA) generativa, impulsionada pelo lançamento em larga escala do ChatGPT pela OpenAI. Essa ferramenta marcou o início de uma nova era no desenvolvimento de software, redefinindo processos e práticas tradicionais.

Nesse contexto desta ``\textit{nova realidade}'', enfrentamos desafios associados a cenários antes \textit{distópicos} ou dignos de obras de ficção científica. A IA tem transformado profundamente o desenvolvimento de software, introduzindo ferramentas que automatizam e aprimoram processos outrora exclusivos de desenvolvedores humanos. Ferramentas como GitHub Copilot, ChatGPT e Claude, baseadas em IA generativa, permitem a geração de código, sugestões em tempo real e até a execução de tarefas complexas a partir de comandos em linguagem natural, os famosos \textbf{prompts}. Esse cenário redefine a interação entre desenvolvedores e sistemas computacionais, criando um ambiente colaborativo entre humanos e máquinas.

\begin{citacao}
``Eu diria que talvez 20\%, 30\% do código que está em nossos repositórios hoje e em alguns de nossos projetos provavelmente foi todo escrito por software'' \citeonline{nadella_2025}.
\end{citacao}

A adoção crescentemente acelerada dessas ferramentas por empresas e comunidades de desenvolvedores levanta questões sobre produtividade, qualidade, segurança e o papel do desenvolvedor no processo de produção de código. Relatórios corporativos, como os da Microsoft e GitHub, indicam que uma proporção significativa de código já é gerada por IA, evidenciando a escala dessa transformação. Contudo, os impactos reais dessas tecnologias, incluindo benefícios, desafios e riscos, ainda carecem de análises sistemáticas baseadas em dados públicos.

\begin{citacao}
``Como em todas as revoluções tecnológicas, espero que haja um impacto significativo nos empregos, mas é muito difícil prever exatamente como será esse impacto'' \citeonline{altman_2023}.
\end{citacao}

Essa pesquisa é motivada pela necessidade de compreender como a IA está moldando o desenvolvimento de software, tanto em termos técnicos quanto nas dinâmicas de trabalho. A disponibilidade crescente, embora limitada, de dados públicos, como relatórios de empresas e repositórios de código, parece oferecer uma oportunidade para investigar essas mudanças, contribuindo para o debate acadêmico e profissional sobre o futuro da programação e das dinâmicas que a envolvem.

\subsection{Motivação dos autores}

Como estudantes de Engenharia de Software, somos confrontados com um momento de ruptura tecnológica que redefine nossa profissão e o meio em que ela está inserida. A integração acelerada da IA no desenvolvimento de software nos inspira a explorar como essas ferramentas estão transformando os processos de produção de código e o papel dos desenvolvedores.

Nossa motivação surge de dúvidas e receios que se intensificam a cada avanço da IA. Figuras como Mark Zuckerberg afirmam que, em breve, desenvolvedores serão substituídos por IAs capazes de gerar código.

\begin{citacao}
``Nos próximos 18 meses, a maior parte da codificação será feita por IA, e ela será melhor que o trabalho da maioria dos engenheiros plenos'' \citeonline{zuckerberg_2025}.
\end{citacao}

Quando figuras públicas como Mark Zuckerberg, que já esteve no papel de desenvolvedor e hoje é empresário de uma das maiores empresas do meio tecnológico, a \textit{Meta}, fazem esse tipo de declaração, somos levados a preocupações específicas.

Ainda nesse contexto, tornam-se recorrentes notícias e relatos acerca de como profissionais da área de software lidam com as ferramentas de geração de código. Desde recortes que incluem trabalhadores recém-formados ou não formados que entraram no mercado há pouco tempo, até profissionais mais experientes com anos de carreira, já podem ser identificadas diferentes dificuldades e estratégias empregadas no uso dos recursos de IA generativa por esses setores \citeonline{g1_2024}.

% Placeholder: Parágrafo sobre estratégias de codificação usando IA
% Citar: https://www.unite.ai/pt/codifica%C3%A7%C3%A3o-de-vibra%C3%A7%C3%A3o-como-a-IA-est%C3%A1-mudando-o-desenvolvimento-de-software-para-sempre/
% https://www.terra.com.br/byte/os-jovens-programadores-ja-nao-sabem-programar-a-ia-esta-provocando-a-mesma-revolucao-que-a-calculadora-causou-ha-meio-seculo,ddb73b03978b8a4747ab9361465725861a2ycg81.html
% Desenvolver com falta de definição das melhores práticas
% Citar: https://about.gitlab.com/pt-br/the-source/ai/6-strategies-to-help-developers-accelerate-ai-adoption/

% Placeholder: Parágrafo relacionado à segurança de dados, informações corporativas sensíveis e confiabilidade
% Citar: https://epocanegocios.globo.com/inteligencia-artificial/noticia/2025/05/quem-sera-responsabilizado-quando-os-agentes-de-ia-errarem.ghtml

Essas incertezas nos levam a uma pergunta central: \textit{quais são os reais impactos da Inteligência Artificial na produção de código e como eles afetam o futuro do desenvolvimento de software?}

Essa questão nos motivou a propor uma análise crítica dos efeitos da IA, buscando não apenas mapear suas aplicações, mas também entender suas implicações para a produtividade, qualidade e relações de trabalho. Este trabalho reflete nosso compromisso em contribuir com reflexões fundamentadas para a comunidade técnica e acadêmica, preparando-nos para atuar em um mercado em constante evolução.

\chapter{Problema}\label{chapter:problema}

A rápida adoção de ferramentas de IA no desenvolvimento de software não tem sido suficientemente acompanhada por estudos sistemáticos que avaliem seus impactos reais no processo de produção de código, conforme dados públicos disponíveis. Apesar dos benefícios prometidos, como maior produtividade e automação, permanecem incertezas sobre a qualidade, segurança, manutenibilidade e as implicações éticas e profissionais dos artefatos gerados, bem como inseguranças relacionadas ao futuro dos profissionais, demandando uma investigação baseada em métricas e gerada a partir de dados públicos ou, em alguns casos, privados, para estudos específicos.

\chapter{Objetivos}\label{chapter:objetivos}

O objetivo geral deste trabalho é analisar os impactos do uso de ferramentas de Inteligência Artificial no processo de produção de código, com base em dados públicos e métricas definidas. Os objetivos específicos são:

\begin{itemize}
    \item Avaliar as principais ferramentas de IA utilizadas na produção de código.
    \item Mapear dados públicos e estudos de caso sobre o uso dessas ferramentas.
    \item Analisar os efeitos na produtividade, qualidade e segurança do código gerado.
    \item Investigar as transformações nas relações de trabalho e perspectivas dos desenvolvedores.
    \item Propor um panorama dos possíveis impactos futuros da IA no desenvolvimento de software.
\end{itemize}

\chapter{Organização do Trabalho}\label{chapter:organizacao}

Este trabalho está estruturado em cinco capítulos. A \hyperref[chapter:introducao]{Introdução} apresenta o contexto e a motivação da pesquisa. O \hyperref[chapter:referencial]{Referencial Teórico} reúne conceitos, ferramentas e estudos relevantes sobre IA e produção de código. A \hyperref[chapter:metodologia]{Metodologia} detalha os métodos de pesquisa, análise e critérios utilizados. Os \hyperref[chapter:resultados]{Resultados} apresentam os dados coletados e suas interpretações. As \hyperref[chapter:conclusao]{Considerações Finais} discutem as conclusões, limitações e sugestões para pesquisas futuras.