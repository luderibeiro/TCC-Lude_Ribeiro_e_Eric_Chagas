% \begin{anexosenv}

% \chapter{Máquina de estados em \textit{GDScript}}
% \begin{figure}[h]
%     \caption{Máquina de estados em \textit{GDScript}}
%     \begin{small}
%     \begin{verbatim}
%     extends Node
%     class_name StateMachine

%     @export var initial_state: State

%     var current_state: State
%     var states: Dictionary = {}

%     func _ready():
% 	    for child in get_children():
% 		        if child is State:
% 			             states[child.name.to_lower()] = child
% 			             child.Transition.connect(on_child_transition)
		
% 		        if initial_state:
% 			             initial_state.Enter()
% 			             current_state = initial_state

%     func _process(delta: float):
% 	       if current_state:
% 		          current_state.Update(delta)

%     func _physics_process(delta: float):
% 	    if current_state:
% 		          current_state.Physics_Update(delta)

%     func on_child_transition(state: State, new_state_name: String):
% 	    if state.name.to_lower() != current_state.name.to_lower():
% 		          return

% 	    var new_state = states.get(new_state_name.to_lower())
%         if !new_state:
%             return
	
%         if current_state:
%             current_state.Exit()

%             new_state.Enter()
%             current_state = new_state
%     \end{verbatim}
%     \end{small}
% \end{figure}

% \chapter{Template de um estado em \textit{GDScript}}
% \begin{figure}[h]
%     \caption{\textit{Template} de um estado em \textit{GDScript}}
%     \begin{verbatim}
%         extends Node
%         class_name State
    
%         signal Transition
    
%         func Enter():
%         	   pass
%         func Exit():
%         	   pass
%         func Update(_delta: float):
%         	   pass
%         func Physics_Update(_delta: float):
%         	   pass
%     \end{verbatim}
% \end{figure}

% \end{anexosenv}

% \section{Cronograma Detalhado}
% A Tabela \ref{tab:cronograma_detalhado} apresenta o cronograma detalhado do projeto, destacando as atividades principais e seus respectivos períodos de execução.

% \begin{landscape}
% \begin{table}[htbp]
%     \centering
%     \caption{Cronograma Detalhado do Projeto}
%     \label{tab:cronograma_detalhado}
%     \begin{tabular}{|l|c|c|c|c|c|c|c|c|c|c|c|c|}
%         \hline
%         \textbf{Atividade} & \textbf{Jan} & \textbf{Feb} & \textbf{Mar} & \textbf{Apr} & \textbf{May} & \textbf{Jun} & \textbf{Jul} & \textbf{Aug} & \textbf{Sep} & \textbf{Oct} & \textbf{Nov} & \textbf{Dec} \\
%         \hline
%         Pesquisa Inicial & X & X &   &   &   &   &   &   &   &   &   &   \\
%         \hline
%         Desenvolvimento &   &   & X & X & X & X &   &   &   &   &   &   \\
%         \hline
%         Testes &   &   &   &   & X & X & X &   &   &   &   &   \\
%         \hline
%         Análise de Dados &   &   &   &   &   & X & X & X &   &   &   &   \\
%         \hline
%         Redação do Relatório &   &   &   &   &   &   & X & X & X & X &   &   \\
%         \hline
%         Revisão e Submissão &   &   &   &   &   &   &   &   & X & X & X & X \\
%         \hline
%     \end{tabular}
% \end{table}
% \end{landscape} 