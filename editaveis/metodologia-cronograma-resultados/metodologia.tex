\chapter[Metodologia]{Metodologia} \label{chapter:Metodologia}

\nocite{*}

Este capítulo detalha o desenho da pesquisa e o plano metodológico adotado para atingir os objetivos propostos. A abordagem combina uma fundamentação teórica, construída a partir da revisão de literatura, com um estudo de caso prático de caráter experimental. Serão descritos os procedimentos da pesquisa, o desenho do experimento, os critérios para a seleção do objeto de estudo, as ferramentas utilizadas, as métricas de avaliação e o protocolo de execução que guia a coleta e a análise dos dados. Ao final, são discutidas as limitações inerentes a este desenho metodológico.

\section*{Nota sobre o uso de Inteligência Artificial}
Em conformidade com as diretrizes acadêmicas de integridade e transparência, declara-se que ferramentas de Inteligência Artificial Generativa (\textbf{ChatGPT, GitHub Copilot e Google Gemini}) foram utilizadas na elaboração deste trabalho. Seu uso restringiu-se à revisão gramatical, sugestão de reestruturação de parágrafos para clareza, aprimoramento textual, geração de \textit{snippets} de código demonstrativos e auxílio na estruturação de ideias. Toda a concepção teórica, a análise crítica dos dados, a discussão dos resultados e as conclusões apresentadas são de autoria intelectual original dos pesquisadores humanos.

\section{Abordagem da Pesquisa}
Este trabalho adota uma abordagem de métodos mistos. A primeira fase, de natureza qualitativa, consiste em uma \textbf{Revisão da Literatura} para consolidar o entendimento sobre os impactos de ferramentas de IA no desenvolvimento de software. A segunda fase, de natureza quantitativa, é um \textbf{Estudo de Caso Empírico} com desenho experimental, que busca gerar evidências concretas sobre os efeitos da IA na qualidade do código.

A revisão da literatura informa o estudo de caso, especialmente na seleção das métricas relevantes (detalhadas na Seção \ref{sec:metricas_analise}), enquanto o experimento prático oferece os dados que serão discutidos à luz do conhecimento teórico estabelecido.

\section{Fase 1: Revisão da Literatura}
A construção da base teórica desta pesquisa foi realizada por meio de uma revisão focada em cinco artigos acadêmicos recentes, selecionados por sua relevância direta com os objetivos deste trabalho. O propósito foi estabelecer um framework conceitual para guiar o estudo empírico, definindo quais dimensões de impacto — qualidade, produtividade, segurança e experiência do desenvolvedor — são mais proeminentes na literatura atual.

% A tabela de artigos (Tabela 1) permanece aqui.
\begin{table}[H]
    \centering
    \caption{Comparação entre os artigos selecionados para a fundamentação teórica.}
    \label{tab:comparacao-artigos}
    % ... (Conteúdo da sua Tabela 1 permanece inalterado) ...
\end{table}

\section{Cenário Atual e Seleção das Ferramentas de IA}
Embora o mercado de Inteligência Artificial Generativa tenha se expandido rapidamente com o surgimento de modelos robustos como o \textbf{Claude 3.5 Sonnet} (Anthropic), conhecido por sua alta capacidade de raciocínio lógico, e o \textbf{Gemini 1.5 Pro} (Google), que oferece integração nativa com o ecossistema Google e ampla janela de contexto, este trabalho optou por focar nas ferramentas da OpenAI e GitHub.

A escolha do \textbf{ChatGPT (modelo GPT-4)} como o "Arquiteto" justifica-se por sua posição consolidada como \textit{benchmark} de mercado, ampla disponibilidade de documentação e capacidade comprovada em tarefas de \textit{Zero-Shot Learning} para design de sistemas.

Para a função de "Copiloto" na codificação, a escolha do \textbf{GitHub Copilot} deve-se à sua integração profunda com o ambiente de desenvolvimento (VS Code) e sua liderança no segmento de \textit{AI Pair Programming}. Diferente de ferramentas baseadas apenas em chat, o Copilot possui acesso ao contexto do \textit{workspace} (arquivos abertos), característica fundamental para a metodologia de refatoração proposta neste estudo.

\section{Fase 2: Estudo de Caso Empírico}
Esta fase consiste na execução do experimento prático para coletar dados quantitativos sobre o impacto do uso das ferramentas GitHub Copilot e ChatGPT no desenvolvimento de software.

\subsection{Desenho do Experimento}
O experimento segue um desenho de \textbf{análise pré e pós-intervenção}, conforme ilustrado na Figura \ref{fig:fluxograma_metodologia}. Esta abordagem foi escolhida por sua capacidade de isolar e quantificar o impacto de uma intervenção específica. O processo envolve três etapas centrais:
\begin{enumerate}
    \item Estabelecimento de uma \textbf{linha de base (baseline)} de qualidade através da medição do estado inicial do sistema;
    \item Execução de uma \textbf{intervenção controlada} de desenvolvimento de novas funcionalidades com assistência integral de IA;
    \item Uma nova medição para \textbf{avaliação comparativa} dos artefatos de código gerados.
\end{enumerate}

% --- INÍCIO DO CÓDIGO DO FLUXOGRAMA (PRESERVADO) ---
\begin{figure}[htbp]
    \centering
    \begin{tikzpicture}[
        node distance=1.5cm and 2cm,
        font=\small,
        fase/.style={rectangle, rounded corners, draw, thick, fill=blue!10, text width=4.5cm, minimum height=1cm, align=center},
        processo/.style={rectangle, draw, thick, fill=green!10, text width=4.5cm, minimum height=1cm, align=center},
        artefato/.style={shape=rectangle, draw, thick, fill=orange!10, text width=3.5cm, align=center, minimum height=1cm},
        ferramenta/.style={shape=ellipse, draw, dashed, fill=gray!10, font=\scriptsize, align=center},
        conector/.style={-Stealth, thick}
    ]

    % FASE I
    \node[fase] (fase1) {\textbf{Fase I: Preparação e Baseline}};
    \node[processo, below=of fase1] (p1_isolar) {1. Definir Baseline do \texttt{django\_base}};
    \node[ferramenta, left=0.3cm of p1_isolar, anchor=east] (git1) {Git};
    \node[processo, below=of p1_isolar] (p2_analisar) {2. Executar Análise Estática Inicial};
    \node[ferramenta, left=0.3cm of p2_analisar, anchor=east] (sonar1) {SonarQube};
    \node[artefato, below=of p2_analisar] (a1_baseline) {Relatório de Baseline};
    
    % FASE II
    \node[fase, below=2cm of a1_baseline] (fase2) {\textbf{Fase II: Execução da Intervenção}};
    \node[processo, below=of fase2] (p3_desenvolver) {3. Refatorar e Incrementar Código};
    \node[ferramenta, left=0.3cm of p3_desenvolver, anchor=east] (ia) {Copilot \&\\ChatGPT};
    \node[artefato, below=of p3_desenvolver] (a2_diario) {Diário de Bordo};

    % --- COLUNA DA DIREITA (Fases III e IV) ---
    \node[fase, right=of fase1] (fase3) {\textbf{Fase III: Validação e Medição Final}};
    \node[processo, below=of fase3] (p4_validar) {4. Validar Código (Testes Unitários)};
    \node[processo, below=of p4_validar] (p5_analisar_final) {5. Executar Análise Estática Final};
    \node[ferramenta, right=0.3cm of p5_analisar_final, anchor=west] (sonar2) {SonarQube};
    \node[artefato, below=of p5_analisar_final] (a3_final) {Relatório Final};
    \node[fase, below=2cm of a3_final] (fase4) {\textbf{Fase IV: Análise Comparativa}};
    \node[processo, below=of fase4] (p6_comparar) {6. Comparar Dados e Calcular Variações};
    \node[artefato, below=of p6_comparar] (a4_resultados) {Resultados e Discussão};

    % --- CONECTORES ---
    \draw[conector] (fase1) -- (p1_isolar);
    \draw[conector] (p1_isolar) -- (p2_analisar);
    \draw[conector] (p2_analisar) -- (a1_baseline);
    \draw[conector] (fase2) -- (p3_desenvolver);
    \draw[conector] (p3_desenvolver) -- (a2_diario);
    \draw[conector] (fase3) -- (p4_validar);
    \draw[conector] (p4_validar) -- (p5_analisar_final);
    \draw[conector] (p5_analisar_final) -- (a3_final);
    \draw[conector] (fase4) -- (p6_comparar);
    \draw[conector] (p6_comparar) -- (a4_resultados);
    \draw[conector, dashed] (a1_baseline) -- (fase2);
    \draw[conector, dashed] (a3_final) -- (fase4);
    \draw[conector, line width=1.5pt] (a2_diario.east) -- ++(1.75,0) |- (fase3.west);
    \draw[conector, gray] (a1_baseline.east) -- ++(1,0) |- (p6_comparar.west);
    \draw[conector, gray] (a3_final.east) -- ++(1,0) |- (p6_comparar.east); 

    \end{tikzpicture}
    \caption{Fluxograma do desenho experimental pré e pós-intervenção, ilustrando o fluxo entre as fases de análise.}
    \label{fig:fluxograma_metodologia}
\end{figure}
% --- FIM DO CÓDIGO DO FLUXOGRAMA ---

\subsection{Evolução do Objeto de Estudo e Histórico}
Durante a fase inicial da pesquisa (TCC 1), o projeto \textit{open-source} \textbf{Zulip} foi selecionado como objeto de estudo para a aplicação das intervenções de IA. No entanto, durante a execução preliminar, identificou-se que a complexidade arquitetural do Zulip e, especificamente, a latência de seu pipeline de CI/CD (com tempos de execução de testes superiores a 30 minutos) inviabilizavam o ciclo rápido de \textit{feedback} necessário para medir a produtividade da IA em tempo real.

Além disso, a análise estática inicial do módulo selecionado no Zulip revelou uma cobertura de testes de apenas 22\%, o que criava uma linha de base ruidosa e dificultava a distinção entre o impacto da intervenção e a dívida técnica preexistente. Devido a esses impedimentos técnicos, o Zulip foi descontinuado como foco da intervenção ativa, sendo mantido neste trabalho apenas como registro das limitações metodológicas encontradas em projetos legados de alta complexidade.

\subsection{Seleção do Novo Objeto de Estudo: \texttt{django\_base}}
Para garantir a viabilidade e o rigor do experimento, o objeto de estudo foi substituído pelo projeto \textit{open-source} \textbf{\texttt{django\_base}} (v2.1.0), um template profissional para projetos Django desenvolvido pelos autores. Esta escolha foi fundamentada em:

\begin{itemize}
    \item \textbf{Linha de Base Robusta:} O projeto apresenta uma \textit{baseline} de alta maturidade (classificação "A" no SonarCloud e 93\% de cobertura de testes). Isso permite que ele sirva como um ambiente experimental controlado, onde o impacto da IA pode ser medido com precisão (sinal claro sem ruído).
    \item \textbf{Controle do Ambiente:} O domínio completo do ambiente elimina barreiras de infraestrutura, permitindo a execução rápida de ciclos de análise (Prompt $\rightarrow$ Código $\rightarrow$ Teste $\rightarrow$ Refatoração).
    \item \textbf{Arquitetura Modular:} A estrutura de Arquitetura Limpa do projeto é ideal para o desenho experimental de "evolução de software", permitindo que novos módulos sejam acoplados sem interferência no código legado.
\end{itemize}

Os resultados apresentados neste trabalho referem-se, portanto, exclusivamente às intervenções realizadas no \texttt{django\_base}.

\subsection{Ambiente e Ferramentas de Apoio}
Foi definido um ambiente de trabalho padronizado para garantir a consistência do experimento.
% ... (lista de ferramentas permanece idêntica) ...

\subsection{Definição do Escopo Experimental (Requisitos)}
\label{sec:requisitos_intervencao}
Para avaliar o impacto da IA em cenários de complexidade distintos, o experimento consiste em duas intervenções de desenvolvimento "greenfield" (código novo). A seguir, são especificados os requisitos para cada intervenção, que formarão o \textit{Product Backlog} do experimento.

\subsubsection{Intervenção A: Módulo de Lógica de Negócios (Carrinho de Compras)}
O objetivo é avaliar a capacidade da IA em lidar com lógica de negócios complexa e gerenciamento de estado.

\paragraph{Histórias de Usuário (Requisitos Funcionais)}
\begin{itemize}
    \item \textbf{HU-A1:} Como um usuário anônimo, eu quero adicionar um produto ao meu carrinho (baseado em sessão), para que eu possa comprá-lo mais tarde.
    \item \textbf{HU-A2:} Como um usuário anônimo, eu quero visualizar todos os itens e o valor total do meu carrinho, para que eu possa revisar meu pedido.
    \item \textbf{HU-A3:} Como um usuário anônimo, eu quero atualizar a quantidade de um item no meu carrinho, para que a lógica de cálculo do total seja re-executada.
    \item \textbf{HU-A4:} Como um usuário anônimo, eu quero remover um item do meu carrinho, para que eu não o compre.
\end{itemize}

\paragraph{Requisitos Não Funcionais}
\begin{itemize}
    \item \textbf{RNF-A1 (Testabilidade):} A lógica de negócios (cálculo de subtotal, total, adição e remoção) deve ser validada por testes unitários.
    \item \textbf{RNF-A2 (Manutenibilidade):} As funções e métodos do módulo devem manter uma Complexidade Ciclomática baixa (inferior a 10).
    \item \textbf{RNF-A3 (Confiabilidade):} A intervenção não deve introduzir nenhum `Bug` (classificação do SonarQube) no código novo.
\end{itemize}

\subsubsection{Intervenção B: Módulo de Segurança (Login Social OAuth2)}
O objetivo é avaliar a capacidade da IA em implementar um fluxo de segurança, uma área onde a literatura aponta riscos significativos.

\paragraph{Histórias de Usuário (Requisitos Funcionais)}
\begin{itemize}
    \item \textbf{HU-B1:} Como um novo usuário, eu quero me cadastrar/autenticar no sistema usando minha conta do Google, para não precisar criar uma conta local com senha.
    \item \textbf{HU-B2:} Como um usuário existente, eu quero fazer login usando minha conta do Google, para acessar minha conta rapidamente e de forma segura.
\end{itemize}

\paragraph{Requisitos Não Funcionais}
\begin{itemize}
    \item \textbf{RNF-B1 (Segurança):} O fluxo deve implementar o protocolo `Authorization Code Grant` do OAuth2.
    \item \textbf{RNF-B2 (Segurança):} As chaves secretas (API keys, client secrets) não devem estar expostas no código-fonte (\textit{hardcoded}), devendo ser lidas de variáveis de ambiente.
    \item \textbf{RNF-B3 (Segurança):} O código novo gerado não deve conter nenhuma `Vulnerabilidade` ou `Hotspot de Segurança` de criticidade alta ou média (classificação do SonarQube).
\end{itemize}

\subsubsection{Backlog do Produto Experimental}
A Tabela \ref{tab:backlog_experimento} consolida as \textit{features} (épicos) e suas respectivas histórias de usuário.

\begin{table}[htbp]
    \centering
    \caption{Backlog do Produto para o experimento de intervenção.}
    \label{tab:backlog_experimento}
    \begin{tabularx}{\textwidth}{l l X}
        \toprule
        \textbf{Feature (Épico)} & \textbf{ID da História} & \textbf{Descrição (História de Usuário)} \\
        \midrule
        \multirow{4}{*}{\parbox{3.5cm}{Gerenciamento de Carrinho de Compras}} 
        & HU-A1 & Adicionar produto ao carrinho (sessão). \\
        & HU-A2 & Visualizar carrinho e total. \\
        & HU-A3 & Atualizar quantidade de item no carrinho. \\
        & HU-A4 & Remover item do carrinho. \\
        \midrule
        \multirow{2}{*}{\parbox{3.5cm}{Autenticação via Google (OAuth2)}} 
        & HU-B1 & Cadastrar-se no sistema com conta Google. \\
        & HU-B2 & Fazer login no sistema com conta Google. \\
        \bottomrule
    \end{tabularx}
    \legend{Fonte: os Autores.}
\end{table}

\subsection{Arquitetura da Aplicação}
Foi definida e estruturada, com o auxílio do assistente de IA ChatGPT, a arquitetura final para a aplicação. O objetivo desse processo é entender como o agente de IA interpreta os requisitos e contribui para a definição arquitetural.

\subsubsection{Arquitetura sugerida pelo Agente}
A arquitetura para a intervenção no projeto \texttt{django\_base} foi desenhada para otimizar a testabilidade e a segurança. A estrutura segue o padrão nativo do Django, utilizando a separação por *Apps* de domínio e introduzindo uma \textbf{Camada de Serviço} para isolar a lógica de negócios.

\paragraph{Estrutura e Isolamento por Apps:}
O projeto será composto pelos seguintes Apps, garantindo o Princípio da Responsabilidade Única (SRP):
\begin{itemize}
    \item \texttt{shopping\_cart}: App responsável pela Intervenção A (Carrinho de Compras).
    \item \texttt{accounts}: App responsável pela Intervenção B (Segurança/OAuth2).
\end{itemize}

\paragraph{Camada de Serviço (Intervenção A):}
Para cumprir o RNF-A1 (Testabilidade) e RNF-A2 (Complexidade), toda a lógica complexa de gerenciamento é encapsulada em uma camada de serviço (\texttt{shopping\_cart/services.py}).
\begin{itemize}
    \item A \textbf{Camada de Serviço} manipula o estado do carrinho.
    \item As Views HTTP apenas delegam chamadas a esta camada e lidam com a resposta.
\end{itemize}

\paragraph{Segurança e Configuração (Intervenção B):}
Para atender aos requisitos de segurança (RNF-B1 a RNF-B3):
\begin{itemize}
    \item \textbf{Autenticação:} Utilização de bibliotecas consolidadas para implementar o fluxo \texttt{Authorization Code Grant}.
    \item \textbf{Segredos:} Todas as chaves devem ser lidas via \textbf{variáveis de ambiente} (\texttt{.env}), prevenindo a exposição no código-fonte.
\end{itemize}

\subsubsection{Desenho da Arquitetura Definida}
O desenho da arquitetura pode ser visualizado na Figura \ref{fig:diagrama_arquitetura}.

\begin{figure}[htbp]
    \centering
    \caption{Diagrama de Arquitetura de Software.}
    % Substitua pelo caminho correto da imagem da arquitetura
    \includegraphics[width=\textwidth]{figuras/TCC_Architecture_V1.png}
    \label{fig:diagrama_arquitetura}
    \legend{Fonte: os Autores.}
\end{figure}

O diagrama ilustra o isolamento dos domínios da intervenção (Carrinho e Segurança) e a separação de responsabilidades (SRP) dentro do projeto \texttt{django\_base}.

\paragraph{Níveis de Abstração e Componentes:}
\begin{enumerate}
    \item \textbf{Contêiner Principal (\texttt{django\_base}):} Representa o limite da aplicação.
    \item \textbf{Módulos Internos:}
    \begin{itemize}
        \item \textbf{Views/Interface:} Camada de entrada/saída que recebe requisições HTTP.
        \item \textbf{Lógica de Negócios (Serviços):} Contém as classes puras do \texttt{shopping\_cart/services.py}. É isolada e altamente testável.
        \item \textbf{Autenticação/Integração:} Lida com o fluxo de segurança do \texttt{accounts} App.
    \end{itemize}
    \item \textbf{Serviços Externos:} Componentes de infraestrutura necessários.
\end{enumerate}

\paragraph{Comunicações Críticas:}
\begin{itemize}
    \item \textbf{Fluxo Carrinho (Intervenção A):} A camada de Interface delega à Lógica de Negócios, que se comunica com o Sistema de Sessão.
    \item \textbf{Fluxo Segurança (Intervenção B):} A camada de Autenticação se comunica com o provedor OAuth2. A Configuração obtém chaves estritamente das Variáveis de Ambiente, cumprindo o RNF-B2.
\end{itemize}

\subsection{Definição das Métricas de Análise} \label{sec:metricas_analise}
% (Sem alteração nesta seção)
A seleção das métricas foi um processo deliberado...
% ... (Texto das métricas permanece idêntico) ...

\subsection{Critérios de Interpretação dos Resultados} \label{sec:interpretacao_resultados}
% (Sem alteração nesta seção)
% ... (Tabela de interpretação permanece idêntica) ...

\subsection{Protocolo de Execução do Experimento}
O protocolo experimental foi revisado para se adequar ao novo objeto de estudo. O processo será dividido em duas intervenções paralelas, executadas por pesquisadores distintos.

\subsubsection{Intervenção A: Módulo de Lógica de Negócios (Carrinho de Compras)}
\begin{itemize}
    \item \textbf{Pesquisador Responsável:} Lude Ribeiro.
    \item \textbf{Objetivo:} Implementar um novo módulo de "Carrinho de Compras" anônimo.
    \item \textbf{Hipótese a Validar:} Avaliar a capacidade da IA (Copilot e ChatGPT) em lidar com lógica de negócios complexa, gerenciamento de estado e geração de testes.
    \item \textbf{Indicadores-Chave:} Complexidade Ciclomática e Cobertura de Testes.
\end{itemize}

\subsubsection{Intervenção B: Módulo de Segurança (Login Social OAuth2)}
\begin{itemize}
    \item \textbf{Pesquisador Responsável:} Eric Chagas.
    \item \textbf{Objetivo:} Implementar a integração com um provedor de "Login Social" via OAuth2.
    \item \textbf{Hipótese a Validar:} Avaliar a eficácia da IA na implementação de fluxos de segurança, verificando a inserção de vulnerabilidades.
    \item \textbf{Indicadores-Chave:} Vulnerabilidades e Hotspots de Segurança (CodeQL/Sonar).
\end{itemize}

\subsubsection*{Fases de Execução (Aplicadas a cada Intervenção)}
Cada intervenção seguirá rigorosamente o mesmo protocolo de 4 fases em \textit{feature branches} Git separadas:

\subsubsection*{Fase I: Preparação}
\begin{enumerate}
    \item \textbf{Isolamento do Código-Fonte:} Criação de uma \textit{feature branch} a partir da \textit{main} (baseline).
    \item \textbf{Planejamento da Intervenção:} Definição dos requisitos específicos.
\end{enumerate}

\subsubsection*{Fase II: Execução da Intervenção (Desenvolvimento Assistido)}
\begin{enumerate}
    \setcounter{enumi}{2}
    \item \textbf{Desenvolvimento Assistido por IA:} Realização da codificação com o auxílio integral do GitHub Copilot e ChatGPT.
    \item \textbf{Registro Sistemático (Diário de Bordo):} Manutenção de um diário de bordo (disponível no Apêndice B) registrando os \textit{prompts}, sugestões acatadas/rejeitadas e justificativas.
\end{enumerate}

\subsubsection*{Fase III: Validação e Medição Final (Pós-Intervenção)}
\begin{enumerate}
    \setcounter{enumi}{4}
    \item \textbf{Validação Funcional:} Verificação de que o código compila e os testes passam.
    \item \textbf{Execução da Análise Estática:} Execução do SonarQube e `pytest-cov` na \textit{feature branch}.
    \item \textbf{Coleta de Dados Finais:} Arquivamento dos relatórios de qualidade.
\end{enumerate}

\subsubsection*{Fase IV: Análise Comparativa dos Resultados}
\begin{enumerate}
    \setcounter{enumi}{7}
    \item \textbf{Tabulação dos Dados:} Organização dos dados pós-intervenção.
    \item \textbf{Análise Crítica:} Avaliação da qualidade intrínseca do código gerado "do zero" pela IA, utilizando a \textit{baseline} de 93\% de cobertura como padrão-ouro.
    \item \textbf{Interpretação:} Análise dos resultados quantitativos, correlacionando-os com o Diário de Bordo.
\end{enumerate}

\subsection{Limitações da Metodologia}
\label{sec:limitacoes}
É imperativo reconhecer as limitações deste novo desenho de pesquisa.
\begin{itemize}
    \item \textbf{Generalização dos Resultados:} Por se tratar de um estudo de caso focado em um único projeto (\texttt{django\_base}), os resultados representam uma evidência pontual e aprofundada, não uma conclusão universal para todos os contextos de software.
    \item \textbf{Viés de Autoria e Familiaridade:} A principal limitação é o uso de um projeto de autoria de um dos pesquisadores. Isso introduz um viés de familiaridade que pode otimizar a interação com o código. No entanto, esta limitação é mitigada pela adoção de uma linha de base de alta maturidade (padrão rigoroso a ser mantido) e pelo uso de ferramentas de análise estática (SonarQube) como "árbitro" imparcial da qualidade.
\end{itemize}