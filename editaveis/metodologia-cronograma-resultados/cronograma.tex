\chapter{Cronograma de Execução e Gestão de Riscos}
\label{chap:cronograma}

Este capítulo apresenta o planejamento temporal e o registro da execução das atividades do TCC 2. Diferente de um cronograma linear tradicional, este plano de trabalho reflete a \textbf{natureza adaptativa} da pesquisa experimental. O cronograma foi reestruturado durante o curso do semestre para acomodar a decisão estratégica de substituição do objeto de estudo (Pivot), mitigando os riscos técnicos que ameaçavam a viabilidade do prazo final.

O cronograma final consolida-se em um regime de execução intensiva (\textit{Sprint}), visando a entrega do documento completo em \textbf{14 de Dezembro} e a defesa na data prevista de \textbf{16 de Dezembro}.

\section{Detalhamento das Fases e Adaptação do Plano}

A execução do projeto foi segmentada em quatro fases distintas. A Tabela \ref{tab:cronograma_tcc2_parte1} detalha a transição crítica da Fase 1 para a Fase 2, onde ocorreu a mudança de escopo técnico, e a Tabela \ref{tab:cronograma_tcc2_parte2} apresenta o planejamento final de escrita e defesa.

\begin{landscape}
    \begin{table}[htbp]
        \centering
        \caption{Cronograma Executivo TCC 2 - Fase Exploratória e Pivot (Ago/25 a Nov/25)}
        \label{tab:cronograma_tcc2_parte1}
        \begingroup
        \scriptsize
        % Ajustei as larguras para evitar Overfull \hbox
        \setlength{\tabcolsep}{4pt}
        \renewcommand{\arraystretch}{1.3}
        \begin{tabular}{@{} l p{8.5cm} c p{4.0cm} @{}}
            \toprule
            \textbf{Fase} & \textbf{Atividade e Descrição Detalhada} & \textbf{Período (Semanas)} & \textbf{Entregável / Marco} \\
            \midrule
            
            \multirow{3}{*}{\textbf{Fase 1}} & \textbf{Exploração e Baseline (Projeto Zulip):} Configuração de ambiente complexo, tentativa de isolamento de módulos e execução da análise estática inicial. Identificação de bloqueios de infraestrutura. & 18/08 a 31/10 (Sem. 1-10) & Relatório de Impedimentos Técnicos. \\
            \addlinespace[0.5em]
            & \textbf{Análise de Risco e Tomada de Decisão (Pivot):} Avaliação da inviabilidade do ciclo de feedback no Zulip. Seleção do \texttt{django\_base} como novo objeto de estudo. & 01/11 a 07/11 (Sem. 11) & \textbf{Marco de Decisão:} Troca de Objeto de Estudo. \\
            \addlinespace[0.5em]
            & \textbf{Setup do Novo Ambiente (django\_base):} Configuração do pipeline de CI/CD simplificado, Docker Compose e estabelecimento da nova Baseline de qualidade (SonarQube). & 08/11 a 17/11 (Sem. 12-13) & Nova Baseline ("Padrão Ouro") definida. \\
            \midrule
            
            \multirow{4}{*}{\textbf{Fase 2}} & \textbf{Intervenção Assistida por IA (Sprint de Desenvolvimento):} Implementação dos módulos de \textbf{Carrinho de Compras} (Lógica) e \textbf{Autenticação OAuth2} (Segurança) utilizando Copilot e ChatGPT. & \textbf{18/11 a 24/11 (Sem. 14)} & Código funcional, Testes e Diário de Bordo. \\
            \addlinespace[0.5em]
            & \textbf{Validação e Coleta de Métricas:} Execução de testes de regressão, testes de carga e nova análise estática (Pós-Intervenção). Coleta dos relatórios de cobertura. & 25/11 a 27/11 (Sem. 15) & Dados Brutos para Análise. \\
            \addlinespace[0.5em]
            & \textbf{Análise Quantitativa e Visualização:} Processamento dos dados, cálculo de delta (Antes vs. Depois) e geração dos gráficos comparativos. & 28/11 a 30/11 (Sem. 15) & Gráficos do Cap. de Resultados. \\
            \addlinespace[0.5em]
            & \textbf{Escrita do Capítulo de Resultados:} Redação técnica descrevendo os achados, correlacionando métricas com o Diário de Bordo. & 01/12 a 02/12 (Sem. 16) & Capítulo 4 Finalizado. \\
            \bottomrule
        \end{tabular}
        \endgroup
    \end{table}
\end{landscape}

\clearpage

\begin{landscape}
    \begin{table}[htbp]
        \centering
        \caption{Cronograma Executivo TCC 2 - Finalização e Defesa (Dezembro/25)}
        \label{tab:cronograma_tcc2_parte2}
        \begingroup
        \scriptsize
        \setlength{\tabcolsep}{4pt}
        \renewcommand{\arraystretch}{1.3}
        \begin{tabular}{@{} l p{8.5cm} c p{4.0cm} @{}}
            \toprule
            \textbf{Fase} & \textbf{Atividade e Descrição Detalhada} & \textbf{Período (Semanas)} & \textbf{Entregável / Marco} \\
            \midrule
            
            \multirow{3}{*}{\textbf{Fase 3}} & \textbf{Consolidação e Discussão:} Interpretação crítica dos resultados à luz do referencial teórico. Redação das Considerações Finais e Trabalhos Futuros. & 03/12 a 05/12 (Sem. 16) & Texto completo (Draft). \\
            \addlinespace[0.5em]
            & \textbf{Revisão Geral e Normalização ABNT:} Leitura completa, correção de citações, formatação de apêndices (Diário de Bordo e Protocolos) e revisão textual. & 06/12 a \textbf{13/12} (Sem. 17) & Versão Final para Depósito. \\
            \addlinespace[0.5em]
            & \textbf{Entrega Formal do Documento (DATA LIMITE):} Envio do documento final ao orientador e banca examinadora. & \textbf{14/12 (Sem. 17)} & \textbf{MARCO FINAL: ENTREGA.} \\
            \midrule
            
            \multirow{2}{*}{\textbf{Fase 4}} & \textbf{Preparação para Banca:} Elaboração dos slides, roteiro de apresentação e revisão dos principais pontos de arguição. & 10/12 a 15/12 (Sem. 17-18) & Material de Apresentação. \\
            \addlinespace[0.5em]
            & \textbf{Defesa Pública:} Apresentação do trabalho à banca examinadora. & \textbf{16/12 (Confirmado)} (Sem. 18) & \textbf{Conclusão do Curso.} \\
            \bottomrule
        \end{tabular}
        \endgroup
    \end{table}
\end{landscape}

\section{Visualização da Execução (Gantt)}
A Figura \ref{fig:gantt_cronograma} ilustra a distribuição temporal das atividades, destacando a extensão do prazo de escrita e revisão até a entrega final em 14/12.

\begin{figure}[htbp]
    \centering
    % Resizebox garante que o gráfico caiba na largura do texto sem estourar margem
    \resizebox{\textwidth}{!}{%
    \begin{ganttchart}[
        vgrid,
        hgrid,
        x unit=0.5cm,
        y unit chart=0.8cm,
        title height=1,
        bar/.style={fill=blue!50},
        bar incomplete/.style={fill=white},
        milestone/.style={fill=red!70},
        time slot format=simple
    ]{1}{18} 

    % --- CABEÇALHO ---
    \gantttitle{Cronograma TCC 2 - Execução e Pivot (2025)}{18} \\
    \gantttitle{Ago.}{2} 
    \gantttitle{Set.}{4}
    \gantttitle{Out.}{5}  
    \gantttitle{Nov.}{4}
    \gantttitle{Dez.}{3} \\
    \gantttitlelist{1,...,18}{1} \\

    % --- FASE 1: O PIVOT ---
    \ganttgroup[group/.append style={fill=gray!30}]{Fase 1: Exploração e Decisão}{1}{13} \\
    \ganttbar[bar/.append style={fill=gray!50}]{Tentativa Zulip (Impedimentos)}{1}{11} \\
    \ganttmilestone[milestone/.append style={fill=red!90}]{\textbf{Decisão de Pivot}}{11.5} \\
    \ganttbar{Setup Django Base}{12}{13} \\

    % --- FASE 2: EXECUÇÃO ---
    \ganttgroup[group/.append style={fill=blue!20}]{Fase 2: Intervenção (Sprint)}{14}{15} \\
    \ganttbar{\textbf{Dev. Assistido (Cart/Auth)}}{14}{14.5} \\
    \ganttbar{Validação e Métricas}{14.5}{15} \\ 
    \ganttbar{Análise de Dados}{15}{15.5} \\

    % --- FASE 3: ESCRITA E REVISÃO ---
    \ganttgroup[group/.append style={fill=green!30}]{Fase 3: Escrita Final}{16}{17} \\
    \ganttbar{Resultados e Discussão}{16}{16.5} \\
    \ganttbar{Revisão ABNT e Apêndices}{16.5}{17} \\
    \ganttmilestone[milestone/.append style={fill=orange!80}]
    {\textbf{ENTREGA (14/12)}}{17} \\

    % --- FASE 4: BANCA ---
    \ganttgroup[group/.append style={fill=yellow!30}]{Fase 4: Banca}{17}{18} \\
    \ganttbar{Prep. Apresentação}{17}{18} \\
    \ganttmilestone{\textbf{Defesa (16/12)}}{18}
    
    \end{ganttchart}
    }
    \caption{Linha do tempo do projeto, evidenciando o momento de decisão (Pivot) e o cronograma final até a defesa em 16/12.}
    \label{fig:gantt_cronograma}
\end{figure}